\documentclass[11pt]{article}

\usepackage[utf8]{inputenc}
\usepackage{csquotes}
\usepackage[british]{babel}
\usepackage[dvipsnames]{xcolor}
\usepackage{a4wide}
\usepackage{xspace}
\usepackage{amsmath}
\usepackage{graphicx}
\usepackage{algorithm}
\usepackage{algpseudocode}
\usepackage{tikz}
\usetikzlibrary{shapes.misc, positioning}
\usepackage{listings}
\usepackage{verbatim}
\usepackage{listings}
\usepackage{hyperref}
\usepackage{markdown}
\usepackage[dateabbrev=false]{biblatex}
\usepackage{placeins}


\markdownSetup{pipeTables,tableCaptions}
\definecolor{dkgreen}{rgb}{0,0.6,0}
\definecolor{gray}{rgb}{0.5,0.5,0.5}
\definecolor{mauve}{rgb}{0.58,0,0.82}

\addbibresource{report.bib}

\lstset{frame=tb,
  language=Java,
  aboveskip=3mm,
  belowskip=3mm,
  showstringspaces=false,
  columns=flexible,
  basicstyle={\small\ttfamily},
  numbers=left,
  numberstyle=\tiny\color{gray},
  keywordstyle=\color{blue},
  commentstyle=\color{green},
  stringstyle=\color{mauve},
  breaklines=true,
  breakatwhitespace=true,
  tabsize=3
}
\hypersetup{
    colorlinks=true,
    frenchlinks=false,
    bookmarksopen=true,
    breaklinks=true,
    linkcolor=black,
    urlcolor=black,
    citecolor=black,
    pdftitle=DAT251 - Fjellturchallenge - Group 5,
    pdfauthor={Karl Henrik Elg Barlinn, Tobias Eilertsen, Sofia Eika, Maren Holm Hundvin, Torjus Schaathun, Mathias Skallerud Jacobsen}
}

\lstdefinelanguage{kotlin}{
  comment=[l]{//},
  commentstyle={\color{gray}\ttfamily},
  emph={delegate, filter, first, firstOrNull, forEach, lazy, map, mapNotNull, println, return@},
  emphstyle={\color{OrangeRed}},
  identifierstyle=\color{black},
  keywords={abstract, actual, as, as?, break, by, class, companion, continue, data, do, dynamic, else, enum, expect, false, final, for, fun, get, if, import, in, interface, internal, is, null, object, override, package, private, public, return, set, super, suspend, this, throw, true, try, typealias, val, var, vararg, when, where, while,lateinit},
  keywordstyle={\color{blue}\bfseries},
  morecomment=[s]{/*}{*/},
  morestring=[b]",
  morestring=[s]{"""*}{*"""},
  morekeywords={[2]{@Autowired, @RestController, @RequestMapping, @Bean, Array, Byte, Double, Float, Int, Integer, Iterable, Long, Runnable, Short, String, @Valid, @RequestBody,@PostMapping, @GetMapping, @DeleteMapping, @PutMapping}},
  keywordstyle={[2]\color{BurntOrange}\bfseries},
  morekeywords={[3]{accountService, account, window,id,accountRequest}},
  keywordstyle={[3]{\color{DarkOrchid}\bfseries}},
  morekeywords={[4]{AccountRequest, AccountResponse, AccountController,AccountService,AccountCreationRequest}},
  keywordstyle={[4]{\color{Periwinkle}\textbf}},
  sensitive=true,
  stringstyle={\color{ForestGreen}\ttfamily},
}
\begin{document}

\title{title}

\author{Tobias} 

\maketitle

\newpage

\begin{abstract}
veldig abstrakt abstract
\end{abstract}

\tableofcontents

\newpage

\section{ CT model viewer}

Making a VR application to view CT scan models

\subsection{introduction}

When a hospital recieves an injured patient that needs surgery, a CT scan is performed. The CT images are displayed as a 3D model on a computer that helps the medical personel prepare for surgery by visualising bone mass or other tissue. If a surgeon has a good understanding of a problem, it can be possible to perform a surgery that has less risk of complications or requires less resources.

The problem with visualising the model in 2D a limited understanding of what the Bone/tissue actually looks like because of the lack of scale and depth. The solution to visualising the model is to view the model in AR or VR to give medical personel a good feel for what the problem actually looks like.

\subsection{ background }

\subsection{ existing solutions}

\subsubsection { 3d printing }

it is possible to print the 3d model to inspect it physically. This has many advantages, such as the surgeon being able to hold whatever he is operating on, measure the model, try out equipment etc. 
The biggest drawback to this is that the printing process can take more than 24 hours depending on the model, which in some cases is too long. Another drawback is not having any digital tools such as transparency, displaying cross section or being able to change the model in any way.
Another possible usecase for the VR viewer is using it to get a first look at a model, and then deciding if a printed model is necessary, potentially saving resources.

\subsubsection { Other commersial solutions }

medical holodeck https://helse-vest-ikt.no/vr-lab/behandling/medical-holodeck \cite{materialise}




\printbibliography

\end{document}